\documentclass[12pt]{beamer}

\usetheme{Cornell}
\usefonttheme{professionalfonts}
\useinnertheme{rectangles}

\setbeamercovered{transparent}

% Without navigation symbols
\beamertemplatenavigationsymbolsempty

%% Formatierungen
%\usepackage{url}
\usepackage{latexsym}			% schönere Symbole
\usepackage{color}
\usepackage{float}

%% Zeichensätze
\usepackage[utf8]{inputenc}
%\usepackage[iso]{umlaute}
\usepackage{lmodern}

\usepackage{thumbpdf}
\usepackage{wasysym}
%\usepackage{ucs}


\author{Your Name}
\title{Presentation Title}
\subtitle{Presentation Subtitle}
\institute{Department of Computer Science \\
Cornell University}

\date{\now}

\subject{Presentation Title}



  
%% Hyperref
\usepackage{hyperref}

\makeatletter
%\hypersetup{pdftitle={\@title}, pdfauthor={\@author}, linktoc=page, pdfborder={0 0 0 [3 3]}, breaklinks=true, linkbordercolor=unibablauI, menubordercolor=unibablauI, urlbordercolor=unibablauI, citebordercolor=unibablauI, filebordercolor=unibablauI}
\makeatother
\urlstyle{same}

%% Sprache
\usepackage[english]{babel} %german, ngerman
%%\usepackage{abstract}

%% Mathe und Formeln
\usepackage{calc}
\usepackage{amsmath}
\usepackage{amssymb,amsthm,amsfonts}
\usepackage[nice]{nicefrac}
\usepackage{cancel}  %%druchstreichen von Formeln
%
%% Programmieren mit Latex
\usepackage{ifthen}


%\usepackage{dirtree}   %setzen von baumstrukturen

%%%   Fuer anspruchsvolle Tabellen   %%
\usepackage{longtable, colortbl}
\usepackage{multicol, multirow}
%
%%%  Für Grafiken %%
\usepackage{graphicx}
\usepackage{tikz}
%\usepackage{pgfplots}
%\usetikzlibrary{calc,arrows,fit,positioning,trees,snakes,backgrounds,shadows,decorations,decorations.shapes,shapes,patterns,fadings}
%\usepackage[font=footnotesize]{subfig}
%
%%%  Zur Darstellung des Euro-Symbols   %%
\usepackage{eurosym, wasysym}
%\selectlanguage{english}
%
%%%   Fuer Bibtex nach APA Style (American Psychology Association)   %%
%\usepackage[numbers]{natbib}
%


%% Code-Hervorhebung
%% Quellcode
\usepackage{verbatim}            % Quellcode einbinden (\verbatiminput) standardpaket
\usepackage{moreverb} 
%% PseudoCode
\usepackage{algorithm}
\usepackage{algpseudocode}
%%\usepackage{algorithmicx}
%%\floatname{algorithm}{Algorithmus}
\algrenewcommand{\algorithmiccomment}[1]{\hskip1em\textcolor{gray!60}{$\rhd$ #1}}
%%\renewcommand{\listalgorithmname}{Algorithmen}
%%\def\algorithmautorefname{Algorithmus}
%
%%% Code Highlighting
\definecolor{mygray}{gray}{.75}
\usepackage{listings} 
\lstset{numbers=left, numberstyle=\tiny, numbersep=6pt} 
\lstset{language=Python}
\lstset{classoffset=1, morekeywords={mycontext}, keywordstyle=\color{darkgreen}, classoffset=0, keywordstyle=\color{darkblue}}
\lstset{basicstyle=\small, showstringspaces=false, commentstyle=\color{mygray}, breaklines=true, captionpos=b}
\renewcommand{\lstlistingname}{Code-Ausschnitt}
\renewcommand{\lstlistlistingname}{Code-Ausschnitte}
\def\lstlistingautorefname{Code-Ausschnitt}


%%%%%%%%%%%%%%%%%%%%%%%%%%%%%%%%%%%%%%%%%%%%%%%%%%%%%%%%%%%%%%%%%%%%%%%%%%%%%%%%%%%%%%%%%%%%
%%%                                   COMMAND SETUP                                       %%
%%%%%%%%%%%%%%%%%%%%%%%%%%%%%%%%%%%%%%%%%%%%%%%%%%%%%%%%%%%%%%%%%%%%%%%%%%%%%%%%%%%%%%%%%%%%
%\newcommand{\HRule}{\rule{\linewidth}{0.5mm}}
%
%#1 Breite
%#2 Datei (liegt im image Verzeichnis)
%#3 Beschriftung
%#4 Label fuer Referenzierung
\newcommand{\image}[4]{
\begin{figure}[H]
\centering
\includegraphics[width=#1]{images/#2}
\caption{#3}
\label{#4}
\end{figure}
}

%#1 Breite
%#2 Datei (liegt im image Verzeichnis)
%#3 Beschriftung
%#4 Label fuer Referenzierung
\newcommand{\lsimage}[4]{
\begin{landscape}
\begin{figure}
\centering
\includegraphics[height=#1]{images/#2}
\caption{#3
\label{#4}}
\end{figure}
\end{landscape}
}

%#1 Breite
%#2 Datei (liegt im image Verzeichnis)
%#3 Beschriftung
%#4 Label fuer Referenzierung
\newcommand{\pimage}[4]{
\begin{figure}[H]
\centering
\includegraphics[width=#1]{images/#2}
\vspace{-5mm}
\caption{#3
\label{#4}}
\end{figure}
}

%#1 Breite
%#2 Datei (liegt im image Verzeichnis)
%#3 Beschriftung
%#4 Label fuer Referenzierung
\newcommand{\kimage}[3]{
\begin{figure}[H]
\label{#3}
\centering
\includegraphics[width=#1]{image/#2}
\vspace{-5mm}
\end{figure}
}

%#1 Breite
%#2 Datei (liegt im image Verzeichnis)
%#3 Beschriftung
%#4 Label fuer Referenzierung
\newcommand{\lspimage}[4]{
\begin{landscape}
\begin{figure}
\centering
\includegraphics[height=#1]{images/#2}
\vspace{-5mm}
\caption{#3
\label{#4}}
\end{figure}
\end{landscape}
}

\newcommand{\twoimages}[6]{%
\begin{minipage}[t]{.475\textwidth}%
\image{\textwidth}{#1}{#2}{#3}%
\vfill%
\end{minipage}%
\begin{minipage}[t]{.05\textwidth}%
\hfill%
\end{minipage}%
\begin{minipage}[t]{.475\textwidth}%
\image{\textwidth}{#4}{#5}{#6}%
\vfill%
\end{minipage}%
}

\newcommand{\twokimages}[6]{%
\begin{minipage}[t]{.475\textwidth}%
\kimage{\textwidth}{#1}{#2}{#3}%
\vfill%
\end{minipage}%
\begin{minipage}[t]{.05\textwidth}%
\hfill%
\end{minipage}%
\begin{minipage}[t]{.475\textwidth}%
\kimage{\textwidth}{#4}{#5}{#6}%
\vfill%
\end{minipage}%
}

% #1 Caption pic 1
% #2 file pic 1
% #3 label pic 1
% #4 caption pic 2
% #5 file pic 2
% #6 label pic2
% #7 overall caption
% #8 overall label
\newcommand{\twosubs}[8]{
\begin{figure}[H]
\centerline{\subfloat[#1]{\includegraphics[width=.5\textwidth]{images/#2}%
\label{#3}}
\hfil
\subfloat[#4]{\includegraphics[width=.5\textwidth]{images/#5}%
\label{#6}}}
\caption{#7}
\label{#8}
\end{figure}
}

%#1 Datei (liegt im graphic Verzeichnis)
%#2 Beschriftung
%#3 Label fuer Referenzierung
\newcommand{\tikzimage}[3]{%
\begin{figure}[H]%
\centering%
\input{graphic/#1.tikz}%
\caption{#2}%
\label{#3}%
\end{figure}
}

%#1 Datei (liegt im image Verzeichnis)
%#2 Beschriftung
%#3 Label fuer Referenzierung
%#4 Skalierungsfaktor
\newcommand{\scaletikzimage}[4]{%
\begin{figure}[H]%
\centering%
\scalebox{#4}{%
\input{images/#1.tikz}}%
\caption{#2}%
\label{#3}%
\end{figure}
}

%#1 Breite
%#2 Höhe
%#2 Datei (liegt im image Verzeichnis)
%#3 Beschriftung
%#4 Label fuer Referenzierung
\newcommand{\imagebh}[5]{
\begin{figure}[H]
\centering
\includegraphics[width=#1, height=#2]{images/#3}
\caption{#4}
\label{#5}
\end{figure}
}

%#1 Breite
%#2 Datei (liegt im image Verzeichnis)
%#3 zugehörige Bildunterschrift
%#4 Beschriftung
%#5 Label fuer Referenzierung
\newcommand{\mathimage}[5]{
\begin{figure}[H]
\centering
\includegraphics[width=#1]{images/#2}\\
#3
\caption{#4}
\label{#5}
\end{figure}
}

%#1 algorithm name
%#2 algorithm label
%#3 file name in code-folder
\newcommand{\pseudo}[3]{%
\small%
\begin{algorithm}[H]%
\caption{#1}%
\label{#2}%
\input{code/#3.tex}%
\end{algorithm}%
\normalsize%
}

%#1 Text der als todo dargestellt werden soll
%\newcommand{\todo}[1]{
%\begin{quote}
%\textcolor{red}{\textbf{TODO: #1}}
%\end{quote}
%}

\newcommand{\rimage}[2]{
\begin{figure}[H]
\centering
#1
%\caption{#2}
\end{figure}
}

\newcommand \rack {
       {\LARGE $\square$}
}

\newcommand \deftab
{\hspace{1.5cm}\=abcdfffefghijk\hspace{1cm}\=1\hspace{1.5cm}\=1\hspace{1.5cm}\=1\hspace{1.5cm}\=1\hspace{1.5cm}\=1\hspace{1.5cm}\=asdjadj\kill}

\newcommand \einsbisfuenf
{\> {\bf -2} \> {\bf -1} \> {\bf 0} \> {\bf 1} \> {\bf 2} \>}

% #1 videofile
% #2 scalefactor
\newcommand{\video}[2]{%
\includemovie[text={\includegraphics[scale=#2]{praesi/video/#1.png}}, autoplay, mouse=true, repeat=1]{}{}{praesi/video/#1.swf}}
